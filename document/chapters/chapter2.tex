\chapter{Systematic Literature Review}

\label{chapter:systematic_review}

\newcommand{\RQI}{What are the main artificial intelligence and machine learning technologies related to voice user interfaces, specially in Speech Recognition, Speech Synthesis, Natural Language Processing, and Trigger-word detection?}

\newcommand{\RQII}{What is the state-of-the-art and the main services that exist in the market on Voice User Interfaces?}

\newcommand{\RQIII}{oioi}

The development of artificial intelligence technologies in recent times is undeniable. Therefore, it is expected that research and screening of documents that represent this knowledge obtained can lead us to understand how voice assistants work and how to implement a possible template that is capable of receiving different domains/contexts and training the platform to be able to execute commands. in these respective domains, especially, in our case, in human resources.

\section{Research Methodology}

This section explains the strategy by which our research was carried out, including the questions that were defined, the sources used, the search terms, and the inclusion and exclusion criteria. At the end, we present a small summary of what this process was like.

\subsection{Research Questions}

The general objective of our literature review is to understand how voice assistants trained for specific domain contexts work. To be able to answer this question, it is necessary to divide it into others, which will be able to offer a general picture of the current state of the art, and how a solution should be designed. The questions defined in our research are found in table 2.

\begin{tabularx}{0.8\textwidth} {
    | >{\raggedright\arraybackslash}X 
    | >{\centering\arraybackslash}X 
    | >{\raggedleft\arraybackslash}X |}
    \hline
    item 11 & item 12 & item 13 \\
    \hline
    item 21  & item 22  & item 23  \\
    \hline
\end{tabularx}

\begin{table}[h!]
    \centering
    \begin{tabular}{| c | c | c |} 
        \hline
        Research Question ID & Research Question & Research Question Domain \\  
        \hline
        RQ1 & \RQI & e \\ 
        \hline
        RQ2 & 7 & e \\
        \hline
        RQ3 & 545 & ee \\ 
        \hline
    \end{tabular}
    \caption{Research questions}
    \label{table:2}
\end{table}

\subsection{Research Terms}

\subsection{Inclusion and Exclusion Criteria}

\subsection{Data Extraction}

\section{Search Results}

\subsection{\RQI}

Lin

\subsection{\RQII}

In \cite{Hoy201881} the main voice user assistants in the market are Amazon's Alexa, Microsoft's Cortana, Google's Google Assistant, and Apple's Siri. These assistants have mostly common features that includes be able to search for content in the internet, help the user organize its life by setting reminders, tasks or calendar events, and it can be integrate with third-party application to execute commands related to them, like request a trip on Uber, or a food order in a food delivery application, and also control IoT devices like TVs, thermostats, door locks, etc. These services are usually executed through brand-specific smart devices, like Amazon's Echo, and Google's Home.

Those software platforms have issues related to security and privacy. Those devices cannot yet authenticate their users by voice, and considering their privileges, they can be a huge threat. Also, it was comproved that those devices are capable of recognizing ultrasonic frequencies and execute accordingly. Also, many users feel afraid that their data is being constantly recording and being sent to private data centers.

\subsection{\RQIII}

\section{Search Conclusions}

\subsection{\RQI}

\subsection{\RQII}

\subsection{\RQIII}

\section{Additional Resolutions}




